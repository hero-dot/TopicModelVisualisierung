%% Dokumentenklasse (Koma Script) -----------------------------------------
\documentclass[%
   %draft,					% Entwurfsstadium
   final,      			% fertiges Dokument
	 % --- Paper Settings ---
   paper=a4,				% Papierformat
   paper=portrait, 	% landscape
   pagesize=auto, 	% driver
   % --- Base Font Size ---
   fontsize=11pt,		%
	 % --- Koma Script Version ---
   version=last, %
 ]{scrbook} % Classes: scrartcl, scrreprt, scrbook

% Encoding der Dateien (sonst funktionieren Umlaute nicht)
% Fuer Linux -> utf8
% Fuer Windows, alte Linux Distributionen -> latin1

% Empfohlen latin1, da einige Pakete mit utf8 Zeichen nicht
% funktionieren, z.B: listings, soul.
\usepackage[latin1]{inputenc}
\usepackage{minted}
%\usepackage[ansinew]{inputenc}
%\usepackage[utf8]{inputenc}
%\usepackage{ucs}
%\usepackage[utf8x]{inputenc}

%%% Preambel
\input{preambel/settings}
\input{preambel/preambel}
%
%%%% Neue Befehle
\input{macros/newcommands}
\input{macros/tablecommands}

%%% Silbentrennung
\input{preambel/hyphenation}

%% Dokument Beginn %%%%%%%%%%%%%%%%%%%%%%%%%%%%%%%%%%%%%%%%%%%%%%%%%%%%%%%%

% - Deckblatt
% - Zusammenfassung
% - Abstract
% - Inahltsverzeichnis
% - Abbildungsverzeichnis (ggf.)
% - Tabellenverzeichnis (ggf.)
% - Abk�rzungsverzeichnis (ggf.)
% -	Ausarbeitung
%		-	Einleitung
%		- Hauptteil
%		- Schlussfolgerung
% Literaturverzeichnis
% - Anhang (ggf.)
% - Eidesstaatliche Erkl�rung

\begin{document}
% Deckblatt
\begin{titlepage}
\enlargethispage{2\baselineskip}
% Header der Julius-Maximilians-Universit�t W�rzburg	 
\begin{figure}[H]
		\centering
		\includegraphics[width=\textwidth]{images/Titel/uni_header}
\end{figure}

\mbox{}\vspace{4\baselineskip}\\
\sffamily\huge
   \centering
   % Titel
   \titelWA
	 	\vspace{2\baselineskip}\\
\rmfamily\normalsize
		\subjectWA
	\vspace{3\baselineskip}\\	

\begin{center}
		\parbox{0cm}{
			\begin{tabbing}
				\= \hspace*{45mm}\= \kill
				\\
				\>{Eingereicht von:} \> {\authorWA} \\
				\>{Studiengang:} \> {\studyWA} \\
				\>{Matrikelnummer:} \> {\matrikulationWA} \\
				\>{Betreuer:} \> {Prof. Dr. Fr�d�ric Thiesse} \\
				\>{Bearbeitungszeitraum:} \> {\timeWA} \\
			\end{tabbing}
}
\end{center}

  
\begin{figure}[H]
		\centering
		\includegraphics[width=0.30\textwidth]{images/Titel/uni_siegel}
\end{figure}

\rmfamily\normalsize
		Julius-Maximilians-Universit�t W�rzburg\\
		Lehrstuhl f�r Wirtschaftsinformatik und Systementwicklung\\
		Josef-Stangl-Platz 2, 97070 W�rzburg
\end{titlepage}

\frontmatter
\chapter{Zusammenfassung}
\label{sec:zusammenfassung}
Die Zusammenfassung dient dem Leser einen groben �berblick �ber die Inhalte zu gewinnen (kurze Problemstellung, Herangehensweise, L�sungsans�tze und evtl. der Schl�sselerkenntnisse). Der Umfang sollte ca. eine halbe Seite betragen.

Auf der n�chsten Seite soll eine �bersetzung der Zusammenfassung als Abstract in englischer Sprache erfolgen.


\chapter{Abstract}
\label{sec:abstract}

Topic Models are capable of discovering the latent structure of large Corpora. There are various Models that can be used for creating a Topic Model. One of those Models is the Latent Dirichlet Allocation. Topic Model, that have been created with the Latent Dirichlet Allocation are numerical distributions. However, those numerical distributions are hard to interpret, just by exploring the output. A way to visualize those numerical values could enhance the interpretation a lot. There are several methods to visualize Topic Models, which are presented in this article. These can be organized as static and interactive visualization. The described methods for the static visualization are the heatmap and the network. For an interactive visualization of the Topic Model the LDAvis system and a Topic Browser is presented. The last chapter contains the creation of a Topic Model on the 20Newsgroup dataset.  The 20Newsgroup dataset is a popular dataset for experiments in text applications of machine learning techniques. The created Topic Model is then visualized using a heatmap and the LDAvis system. It is possible to improve the interpretability of Topic Models with those visualizations. Nevertheless, the visualization gets confusing for large Topic Models.
\cleardoublepage

%Inahltsverzeichnis
\addcontentsline{toc}{chapter}{Inhaltsverzeichnis}
\tableofcontents

% Abbildungs- und Tabellenverzeichnis
\listoffigures
\listoftables
% Abk�rzungsverzeichnis
\addcontentsline{toc}{chapter}{Abk�rzungsverzeichnis}
%% ftp://ftp.tu-chemnitz.de/pub/tex/macros/latex/contrib/acronym/acronym.pdf
\chapter*{Abk�rzungsverzeichnis}
\begin{acronym}[TDMA] %[l�ngste Abk�rzung]
 \acro{LDA}{Latent Dirichlet Allocation}
 \acro{GSM}{Global System for Mobile Communication}
 \acro{TDMA}{Time Division Multiple Access}
\end{acronym}
Bei der Erstellung des Abbildungsverzeichnisses muss darauf geachtet werden, dass die Abk�rzungen noch in alphabetische Reihenfolge gebracht werden m�ssen. Weiterhin ist zu beachten, dass nur solche Abk�rzungen im Verzeichnis aufgef�hrt werden, die auch im Text verwendet wurden.


% Hauptteil
\mainmatter

%Zeilenabstand
%\onehalfspacing
%\recalctypearea

%Setze Z�hler f�r den Hauptteil neu
\setcounter{page}{1}
\setcounter{chapter}{0}

% Einbinden der einzlenen Kapitel der Ausarbeitung
\chapter{Einleitung}

\paragraph{Einf�hrung in die Problemstellung} 
\label{sec:Einf�hrung in die Problemstellung}
Durch das Internet sind immer mehr Informationen vor allem in Textform verf�gbar. Um diese Informationen zu verarbeiten, verwendet man Verfahren des Topic Modeling. Das Ergebnis dieser Verfahren sind Topic Models. Bei ihnen handelt es sich um eine statistische Auswertung von Textk�rpern. In  der Literatur werden unterschiedliche Verfahren dazu beschrieben.

\paragraph{Motivation und Herleitung des Themas}
\label{sec:Motivation und Herleitung des Themas}
Topic Models lassen sich in der Form, wie sie von den verschiedenen Verfahren nur schwer interpretieren und besitzen unter Umst�nden Topics ohne Aussagegehalt. 

\paragraph{Aufbau der Arbeit}
\label{sec:Aufbau der Arbeit}
Ziel dieser Arbeit ist es, Methoden darzustellen, die Topic Models visualisieren k�nnen und mit deren Hilfe man bestimmte Probleme mit Topic Models beseitigen kann. 
Dazu werden im ersten Teil die g�ngigen Verfahren dargestellt, mit deren Hilfe die Topic Models erstellt werden. Im zweiten Teil wird anschlie�end auf die unterschiedlichen Verfahren zur Visualisierung von Topic Models eingegangen. Das Augenmerk soll dabei auf den unterschiedlichen Perspektiven liegen, die mit dem jeweiligen Verfahren dargestellt wird. Der vierte Teil umfasst einen Vergleich der Verfahren und eine Diskussion ihrer Vor- und Nachteile. Abschlie�end erfolgt noch 
eine beispielhafte Anwendung der Verfahren, die sich als die Verfahren mit den meisten Vorz�gen herausgestellt haben. Dazu wird ein Test Corpus erstellt und auf diesen werden die jeweiligen Verfahren angewendet.   

\paragraph{Sammlung der Ideen}
\label{sec:Sammlung der Ideen}

\begin{itemize}

\item \cite[1]{Chaney.2012} Das Verstehen und organisieren von gro�en Sammlungen von Dokumenten ist zu einer wichtigen T�tigkeit in vielen Bereichen geworden. Viele Sammlungen sind aber nicht sinnvoll organisiert und das Organisieren von Hand ist zu umst�ndlich.

\item \cite[1]{Sievert.2014} In letzter Zeit wurde der Darstellung von Ergebnissen der Latent Dirichlet Allokation (LDA) den Topic Models viel Aufmerksamkeit zu Teil. (Gardner, Chaney, Chuang, Gretarsson) Eine solche Darstellung ist aber nicht einfach umzusetzen, aufgrund der hohen Abstraktion der Ergebnisse. Diese Ergebnisse lassen sich nur vollst�ndig und kompakt darstellen, wenn eine interaktive Darstellungsform gew�hlt wird.

\item \cite[1]{Snyder.2013} Wenn Nutzer Texte analysieren, ben�tigen sie eine intuitive Methode um Texte zu verstehen und zusammenzufassen. Werkzeuge, die dies erm�glichen lassen sich in zwei Kategorien unterteilen. Die eine basiert auf strukturierten Metadaten und die Andere auf Informationen, die aus den Texten extrahiert wurden. 

\item \cite[1]{Blei.2009} Topic Models sind hierarchische Bayes Modelle die einen betrachteten Textkorpus als eine kleine Verteilung von W�rtern darstellen. Die Idee hinter Topic Models ist, sich einen zuf�lligen Prozess vorzustellen aus dem die versteckte Struktur der Themen und die beobachtete Sammlung an Dokumenten zu Stande kommt. Dieser Prozess wird anschlie�end umgekehrt um die posteriore Verteilung der versteckten Themen Struktur zu schlie�en.

\item \cite[1]{Newman.2010} W�hrend viele Nutzer konkrete Informationen zu einem Thema finden m�chten, gibt es eine gro�e Anzahl an Nutzern die  alle Informationen zu einem bestimmten Thema finden und verstehen m�chten und die Reichweite ihrer Suche (in Tiefe und Breite) kennen m�chten. Eine genaue und intuitive Visualisierung der Suchergebnisse kann zu diesem Verst�ndnis beitragen. 

\item[Gedanke] In den letzten Jahren wurden Topic Models immer beliebter. Sie lassen sich jedoch nur schwer interpretieren. Aus diesem Grund wurden unterschiedliche Verfahren zur visuellen Aufbereitung der Topic Models entwickelt. Diese Visualisierungen unterscheiden sich in bestimmten Punkten. 

\item[Gedanke] Das Ergebnis der Verfahren zur Erstellung von Topic Models ist eine Matrix mit den Wahrscheinlichkeiten der einzelnen W�rter. Als solche ist es nur schwer m�glich diese zu interpretieren. 
Beleg?

\item[Gedanke] Was kann die Arbeit und was soll sie nicht k�nnen

\item[Gedanke] Topic Models stellen eine gute M�glichkeit dar gro�e Mengen an Texten zu analysieren und ihnen zu Grunde liegenden Themen herauszufinden. Die Interpretation dieser Themen gestaltet sich jedoch schwierig. Im Rahmen dieser Arbeit soll zun�chst eine einfache Darstellung des Outputs eines Topic Models erfolgen. Um den Output dieser Verfahren einfacher analysieren zu k�nnen wurden in der Literatur Verfahren entwickelt, die dies �bernehmen. 

\end{itemize}




\chapter{Grundlagen}
\label{sec:Grundlagen}
Hier soll jeweils eine kurze Einf�hrung erfolgen, die den Zusammenhang des Kapitels zur Arbeit herstellt.

Generelle Hinweise: Verwenden Sie stets eindeutige Begrifflichkeiten achten Sie auf eine logische Herleitung ihrer  Argumentationen.

\section{Topic Models}
\label{sec:Topic Models}

\subsection{Probabilistic Latent Semantic Analysis}
\label{sec:Probabilistic Latent Semantic Analysis}

\subsection{Latent Dirichlet Allocation}
\label{sec:Latent Dirichlet Allocation}

\section{Darstellungsverfahren}
\label{sec:Darstellungsverfahren}

\subsection{Topic Browser}
\label{sec:Topic Browser}

\subsection{LDAvis}
\label{sec:LDAvis}

\subsection{Termite}
\label{sec:Termite}

\chapter{Visualisierung der Ergebnisse}
\label{sec:Visualisierung der Ergebnisse}



\subsubsection{Vierte Ebene}
\label{sec:vierte_ebene}

\paragraph{F�nfte Ebene}
\label{sec:f�nfte_ebene}



\chapter{Schlussfolgerung}
\label{sec:schlussfolgerung}

Die Visualisierung von Topic Models kann mit verschiedenen Verfahren realisiert werden. Um die Problemstellung bei der Visualisierung und die Eignung der verschiedenen Verfahren zu �berpr�fen, wurden das LDA Modell und seine Dimensionen beschrieben. Im Anschluss wurden Verfahren beschrieben mit denen sich die Dimensionen des LDA Modells darstellen lassen. Dabei wurde auf statische Grafiken, wie Heatmaps und Netzwerke und auf interaktive Darstellungen, wie das LDAvis System und den Topic Browser eingegangen. Die Erstellung eines Topic Models zu dem 20Newsgroup Datensatz und die Visualisierung des Topic Models mit einer Heatmap und dem LDAvis Verfahren ergab die folgende Einsichten in die Problemstellungen bei der Visualisierung von Topic Models:
\\Die Visualisierung mit einer Heatmap eignet sich nur f�r Topic Models von kleinen Sammlungen an Dokumenten oder f�r die Ausschnittsweise Darstellung gr��erer Topic Models. Netzewerke werden mit zunehmender Gr��e des Topic Models  un�bersichtlich. Aufgrund der F�lle an Informationen die ein Topic Model bereith�lt ist eine interaktive Darstellung n�tig. Die Wahl der richtigen interaktiven Darstellung h�ngt von dem Ziel, das mit dem Erstellen des Topic Models verfolgt wird, ab. Ist das Ziel eine gro�e Sammlung an Texten �ber ihre latenten Themen zug�nglich zu machen, eignet sich daf�r ein Topic Browser. Soll die thematische Struktur eines Korpus analysiert werden, bietet sich die Verwendung des LDAvis Systems an. 
\\Die Qualit�t des verwendeten Vokabulars entscheidet �ber die Qualit�t des Topic Models und damit auch �ber die Aussagekraft der Visualisierung. Im vorliegendem Fall enthielten die Topics noch W�rter ohne Aussagekraft. Das f�hrt zu ungenauen Topics. Betrachtet man nun ein Kluster an Topics in der Darstellung mit dem LDAvis System, kann dem Kluster kein eindeutiger Themenbereich zugeteilt werden. 
Die Abbildung aller hier beschriebenen Verfahren wird f�r gro�e Topic Modelle un�bersichtlich. F�r die Heatmap ist das im Vergleich schon bei einer relativ geringen Anzahl an Topics der Fall. Das Netzwerk und das LDAvis System kommen mit einer gr��eren Anzahl an Topics zurecht. Aufgrund der Unterteilung des Topic Browsers in mehrere Ansichten und dem Abbilden einer Teilmenge an Topics und W�rtern kommt der Topic Browser auch mit sehr gro�en Dokumenten zurecht. 

%Anwendung weiterer Verfahren, die hier nur am Rande angesprochen worden sind
%Verwendung von Metadaten


\clearpage

% Literaturverzeichnis
\pagenumbering{Roman}
\bibliography{bibliography/Literatur}
\clearpage

% Anhang
\appendix
\setcounter{page}{1}
\pagenumbering{arabic}
\addchap{Anhang}
\label{sec:anhang}
\refstepcounter{chapter}

\section{Materialien} 
\label{sec:materialien}
Fusce vitae quam eu lacus pulvinar vulputate. Suspendisse potenti. Aliquam imperdiet ornare nibh. Cras molestie tortor non erat. Donec dapibus diam sed mauris laoreet volutpat. Sed at ante id nibh consectetuer convallis. Suspendisse diam tortor, lobortis eget, porttitor sed, molestie sed, nisl.
\clearpage

% Eidesstaatliche Erkl�rung
\input{content/Erklaerung.tex}

%% Dokument Ende %%%%%%%%%%%%%%%%%%%%%%%%%%%%%%%%%%%%%%%%%%%%%%%%%%%%%%%%%%
\end{document}

