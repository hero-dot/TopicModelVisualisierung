\chapter{Einleitung}

Angesichts der zunehmenden Vernetzung und Digitalisierung durch das Internet w�chst die verf�gbare Datenmenge stetig an. Ein gro�er Teil dieser Informationen sind Texte oder Dokumente, die Informationen in geschriebener Form enthalten, wie beispielsweise Webseiten. F�r Menschen ist eine Analyse dieser Dokumente, ohne technische Hilfe nur schwer m�glich. Eine M�glichkeit versteckte Strukturen in diesen Dokumenten sichtbar zu machen ist die Erstellung von Topic Models. \\Ein weit verbreitetes und sehr beliebtes Verfahren f�r die Erstellung eines Topic Models ist die Latent Dirichlet Allocation. Die Idee hinter dem \ac{LDA} ist, dass sich Dokumente aus verschiedenen Themen zusammensetzen. Die Latent Dirichlet Allocation ist ein statistisches Modell, das die thematische Zusammensetzung f�r eine Auswahl an Dokumenten abbilden kann, wobei die W�rter der Dokumente das einzige beobachtbare Merkmal im Modell sind. Die Struktur der Themen ist versteckt\cite[78f.]{Blei.2012}. Mit einem Inferenzverfahren l�sst sich diese versteckte Struktur n�herungsweise bestimmen. Das Ergebnis sind Verteilungen der W�rter f�r jedes Thema, der Sammlung der Dokumente. Bei den Verteilungen handelt es sich um Tabellen mit numerischen Werten, die eine weitere Analyse der Zahlen voraussetzt, um den Sinn in den Topics zu finden. 
Ziel dieser Arbeit ist es, Methoden darzustellen, die Topic Models visualisieren k�nnen. 
\\Dazu erfolgt im Grundlagenteil eine Beschreibung der Latent Dirichlet Allocation. Dort wird das statistische Modell, die Verfahren zur Inferenz und das Ergebnis der Inferenz, die Topic Wort Verteilung erl�utert. Anschlie�end wird die visuelle Darstellung der Topic Models mit statischen Grafiken beschrieben. Neben der Darstellung der Topic Models mit den statischen Grafiken werden Verfahren zur interaktiven Darstellung, dem LDAvis \cite{Sievert.2014} und einem Topic Browser \cite{Chaney.2012} vorgestellt.
Im Anwendungsteil der Arbeit wird ein Topic Model zu dem 20Newsgroup\footnote{\url{http://qwone.com/~jason/20Newsgroups/}} Datensatz erstellt. Das Topic Model wird mit einer statischen Grafik, einer Heatmap und mit dem LDAvis System dargestellt.