\chapter{Einleitung}

\paragraph{Einf�hrung in die Problemstellung} 
\label{sec:Einf�hrung in die Problemstellung}
Durch das Internet sind immer mehr Informationen vor allem in Textform verf�gbar. Um diese Informationen zu verarbeiten, verwendet man Verfahren des Topic Modeling. Das Ergebnis dieser Verfahren sind Topic Models. Bei ihnen handelt es sich um eine statistische Auswertung von Textk�rpern. In  der Literatur werden unterschiedliche Verfahren dazu beschrieben.

\paragraph{Motivation und Herleitung des Themas}
\label{sec:Motivation und Herleitung des Themas}
Topic Models lassen sich in der Form, wie sie von den verschiedenen Verfahren nur schwer interpretieren und besitzen unter Umst�nden Topics ohne Aussagegehalt. 

\paragraph{Aufbau der Arbeit}
\label{sec:Aufbau der Arbeit}
Ziel dieser Arbeit ist es, Methoden darzustellen, die Topic Models visualisieren k�nnen und mit deren Hilfe man bestimmte Probleme mit Topic Models beseitigen kann. 
Dazu werden im ersten Teil die g�ngigen Verfahren dargestellt, mit deren Hilfe die Topic Models erstellt werden. Im zweiten Teil wird anschlie�end auf die unterschiedlichen Verfahren zur Visualisierung von Topic Models eingegangen. Das Augenmerk soll dabei auf den unterschiedlichen Perspektiven liegen, die mit dem jeweiligen Verfahren dargestellt wird. Der vierte Teil umfasst einen Vergleich der Verfahren und eine Diskussion ihrer Vor- und Nachteile. Abschlie�end erfolgt noch 
eine beispielhafte Anwendung der Verfahren, die sich als die Verfahren mit den meisten Vorz�gen herausgestellt haben. Dazu wird ein Test Corpus erstellt und auf diesen werden die jeweiligen Verfahren angewendet.   

\paragraph{Sammlung der Ideen}
\label{sec:Sammlung der Ideen}

\begin{itemize}

\item \cite[1]{Chaney.2012} Das Verstehen und organisieren von gro�en Sammlungen von Dokumenten ist zu einer wichtigen T�tigkeit in vielen Bereichen geworden. Viele Sammlungen sind aber nicht sinnvoll organisiert und das Organisieren von Hand ist zu umst�ndlich.

\item \cite[1]{Sievert.2014} In letzter Zeit wurde der Darstellung von Ergebnissen der Latent Dirichlet Allokation (LDA) den Topic Models viel Aufmerksamkeit zu Teil. (Gardner, Chaney, Chuang, Gretarsson) Eine solche Darstellung ist aber nicht einfach umzusetzen, aufgrund der hohen Abstraktion der Ergebnisse. Diese Ergebnisse lassen sich nur vollst�ndig und kompakt darstellen, wenn eine interaktive Darstellungsform gew�hlt wird.

\item \cite[1]{Snyder.2013} Wenn Nutzer Texte analysieren, ben�tigen sie eine intuitive Methode um Texte zu verstehen und zusammenzufassen. Werkzeuge, die dies erm�glichen lassen sich in zwei Kategorien unterteilen. Die eine basiert auf strukturierten Metadaten und die Andere auf Informationen, die aus den Texten extrahiert wurden. 

\item \cite[1]{Blei.2009} Topic Models sind hierarchische Bayes Modelle die einen betrachteten Textkorpus als eine kleine Verteilung von W�rtern darstellen. Die Idee hinter Topic Models ist, sich einen zuf�lligen Prozess vorzustellen aus dem die versteckte Struktur der Themen und die beobachtete Sammlung an Dokumenten zu Stande kommt. Dieser Prozess wird anschlie�end umgekehrt um die posteriore Verteilung der versteckten Themen Struktur zu schlie�en.

\item \cite[1]{Newman.2010} W�hrend viele Nutzer konkrete Informationen zu einem Thema finden m�chten, gibt es eine gro�e Anzahl an Nutzern die  alle Informationen zu einem bestimmten Thema finden und verstehen m�chten und die Reichweite ihrer Suche (in Tiefe und Breite) kennen m�chten. Eine genaue und intuitive Visualisierung der Suchergebnisse kann zu diesem Verst�ndnis beitragen. 

\item[Gedanke] In den letzten Jahren wurden Topic Models immer beliebter. Sie lassen sich jedoch nur schwer interpretieren. Aus diesem Grund wurden unterschiedliche Verfahren zur visuellen Aufbereitung der Topic Models entwickelt. Diese Visualisierungen unterscheiden sich in bestimmten Punkten. 

\item[Gedanke] Das Ergebnis der Verfahren zur Erstellung von Topic Models ist eine Matrix mit den Wahrscheinlichkeiten der einzelnen W�rter. Als solche ist es nur schwer m�glich diese zu interpretieren. 
Beleg?

\item[Gedanke] Was kann die Arbeit und was soll sie nicht k�nnen

\item[Gedanke] Topic Models stellen eine gute M�glichkeit dar gro�e Mengen an Texten zu analysieren und ihnen zu Grunde liegenden Themen herauszufinden. Die Interpretation dieser Themen gestaltet sich jedoch schwierig. Im Rahmen dieser Arbeit soll zun�chst eine einfache Darstellung des Outputs eines Topic Models erfolgen. Um den Output dieser Verfahren einfacher analysieren zu k�nnen wurden in der Literatur Verfahren entwickelt, die dies �bernehmen. 

\end{itemize}



