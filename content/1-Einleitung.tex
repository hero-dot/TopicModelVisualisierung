\chapter{Einleitung}
Einf�hrung in die Problemstellung
Durch das Internet sind immer mehr Informationen vor allem in Textform verf�gbar. Um diese Informationen zu verarbeiten, verwendet man Verfahren des Topic Modeling. Das Ergebnis dieser Verfahren sind Topic Models. Bei ihnen handelt es sich um eine statistische Auswertung von Textk�rpern. In  der Literatur werden unterschiedliche Verfahren dazu beschrieben.
Motivation und Herleitung des Themas
Topic Models lassen sich in der Form, wie sie von den verschiedenen Verfahren nur schwer interpretieren und besitzen unter Umst�nden Topics ohne Aussagegehalt. 
Aufbau der Arbeit
Ziel dieser Arbeit ist es, Methoden darzustellen, die Topic Models visualisieren k�nnen und mit deren Hilfe man bestimmte Probleme mit Topic Models beseitigen kann. 
Dazu werden im ersten Teil die g�ngigen Verfahren dargestellt, mit deren Hilfe die Topic Models erstellt werden. Im zweiten Teil wird anschlie�end auf die unterschiedlichen Verfahren zur Visualisierung von Topic Models eingegangen. Das Augenmerk soll dabei auf den unterschiedlichen Perspektiven liegen, die mit dem jeweiligen Verfahren dargestellt wird. Der vierte Teil umfasst einen Vergleich der Verfahren und eine Diskussion ihrer Vor- und Nachteile. Abschlie�end erfolgt noch 
eine beispielhafte Anwendung der Verfahren, die sich als die Verfahren mit den meisten Vorz�gen herausgestellt haben. Dazu wird ein Test Corpus erstellt und auf diesen werden die jeweiligen Verfahren angewendet.   
