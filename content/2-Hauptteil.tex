\chapter{Grundlagen}
\label{sec:Grundlagen}
Hier soll jeweils eine kurze Einf�hrung erfolgen, die den Zusammenhang des Kapitels zur Arbeit herstellt.

Generelle Hinweise: Verwenden Sie stets eindeutige Begrifflichkeiten achten Sie auf eine logische Herleitung ihrer  Argumentationen.

\section{Topic Models}
\label{sec:Topic Models}

\subsection{Probabilistic Latent Semantic Analysis}
\label{sec:Probabilistic Latent Semantic Analysis}

\subsection{Latent Dirichlet Allocation}
\label{sec:Latent Dirichlet Allocation}

\section{Darstellungsverfahren}
\label{sec:Darstellungsverfahren}

\subsection{Topic Browser}
\label{sec:Topic Browser}

\subsection{LDAvis}
\label{sec:LDAvis}

\subsection{Termite}
\label{sec:Termite}

\chapter{Visualisierung der Ergebnisse}
\label{sec:Visualisierung der Ergebnisse}



\subsubsection{Vierte Ebene}
\label{sec:vierte_ebene}

\paragraph{F�nfte Ebene}
\label{sec:f�nfte_ebene}


