\chapter{Grundlagen}
\label{sec:Grundlagen}

Um Textstrukturen in gr��eren Sammlungen mit Hilfe von Topic Models visualisieren zu k�nnen ben�tigt man Verfahren, die diese Topic Models von den gesammelten Dokumenten erstellen und geeignete Darstellungsformen f�r diese Topic Models. Wie Topic Models erstellt werden, soll im Rahmen dieser Arbeit nur kurz beschrieben werden. Der Fokus dieser Arbeit liegt auf den Methoden, die Topic Models aufbereiten und darstellen k�nnen. 

Generelle Hinweise: Verwenden Sie stets eindeutige Begrifflichkeiten achten Sie auf eine logische Herleitung ihrer  Argumentationen.

\section{Latent Dirichlet Allocation}
\label{sec:LDA}

In den letzten Jahrzehnten wurden verschiedene Verfahren entwickelt um gro�e Textmengen zu analysieren. Deerwester und Dumais stellten bereits im Jahr 1990 das Indizieren anhand einer latenten semantischen Analyse vor \cite{Deerwester.1990}. Dieses wurde von \cite{Hofmann.1999} zu der probabilistischen latenten semantischen Indizierung erweitert. 
\section{Darstellungsverfahren}
\label{sec:Darstellungsverfahren}

\subsection{Graphische Darstellung der Topic Models}
\label{sec:graphische_darstellung}

\subsection{Topic Browser}
\label{sec:Topic Browser}

\subsection{LDAvis}
\label{sec:LDAvis}

\chapter{Anwendung}
\label{sec:Anwendung}

\section{Topic Models mit dem LDA Verfahren}
\label{sec:ergebnisse_des_LDA}

\section{Visualisierung der Topic Models}
\label{sec:visualisierung_topicmodels}

\subsection{Topic Models als Heatmap}
\label{sec:heatmap}

\subsection{Topic Models als Netzwerk}
\label{sec:networks}

\section{Erstellung eines Topic Browsers}
\label{sec:erstellung_TopicBrowser}

\section{Visualisierung mit LDAvis}
\label{sec:anwendung_LDAvis}

\subsubsection{Vierte Ebene}
\label{sec:vierte_ebene}

\paragraph{F�nfte Ebene}
\label{sec:f�nfte_ebene}


