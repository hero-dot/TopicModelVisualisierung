\chapter{Grundlagen}
\label{sec:Grundlagen}

Hier soll jeweils eine kurze Einf�hrung erfolgen, die den Zusammenhang des Kapitels zur Arbeit herstellt.

Generelle Hinweise: Verwenden Sie stets eindeutige Begrifflichkeiten achten Sie auf eine logische Herleitung ihrer  Argumentationen.

\section{Latent Dirichlet Allocation}
\label{sec:LDA}

\section{Darstellungsverfahren}
\label{sec:Darstellungsverfahren}

\subsection{Graphische Darstellung der Topic Models}
\label{sec:graphische_darstellung}

\subsection{Topic Browser}
\label{sec:Topic Browser}

\subsection{LDAvis}
\label{sec:LDAvis}

\chapter{Anwendung}
\label{sec:Anwendung}

\section{Topic Models mit dem LDA Verfahren}
\label{sec:ergebnisse_des_LDA}

\section{Visualisierung der Topic Models}
\label{sec:visualisierung_topicmodels}

\subsection{Topic Models als Heatmap}
\label{sec:heatmap}

\subsection{Topic Models als Netzwerk}
\label{sec:networks}

\section{Erstellung eines Topic Browsers}
\label{sec:erstellung_TopicBrowser}

\section{Visualisierung mit LDAvis}
\label{sec:anwendung_LDAvis}

\subsubsection{Vierte Ebene}
\label{sec:vierte_ebene}

\paragraph{F�nfte Ebene}
\label{sec:f�nfte_ebene}


