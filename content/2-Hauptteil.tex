\chapter{Grundlagen}
\label{sec:Grundlagen}

Um Textstrukturen in gr��eren Sammlungen mit Hilfe von Topic Models visualisieren zu k�nnen ben�tigt man Verfahren, die diese Topic Models von den gesammelten Dokumenten erstellen und geeignete Darstellungsformen f�r diese Topic Models. Wie Topic Models erstellt werden, soll im Rahmen dieser Arbeit nur kurz beschrieben werden. Der Fokus dieser Arbeit liegt auf den Methoden, die Topic Models aufbereiten und darstellen k�nnen. 


\section{Latent Dirichlet Allocation}
\label{sec:LDA}

\subsection{Topic Model Verfahren im �berblick}
\label{sec:�berblick}
Das erste Verfahren f�r eine Abstraktion von Texten war das Latent semantic Analysis von Deerwester und Dumais\cite{Deerwester.1990}. Darauf aufbauend wurde das Verfahren probabilistic Latent Semantic Indexing von Hofmann eingef�hrt\cite{Hofmann.1999}. 
Aufgrund einiger Unzul�nglichkeiten in dem Verfahren wurde von Blei das Verfahren der Latent Dirichlet Allocation vorgestellt\cite{Blei.2003}. 

\subsection{Vorgehen bei der Latent Dirichlet Allocation}
\label{sec:Vorgehen}

Beschreibung des LDA Verfahrens im Detail. 
Eingehen auf die Verfahren von Steyvers und Griffiths\cite{Griffiths.2002,Griffiths.2004,Griffiths.2003}

Worum es bei dem Artikel geht.Wieso ich LDA verwende \cite[6]{Chang.2009}
 
\section{Darstellungsverfahren}
\label{sec:Darstellungsverfahren}

\subsection{Graphische Darstellung der Topic Models}
\label{sec:graphische_darstellung}

\subsection{Topic Browser}
\label{sec:Topic Browser}

\subsection{LDAvis}
\label{sec:LDAvis}



\chapter{Anwendung}
\label{sec:Anwendung}

\section{Topic Models mit dem LDA Verfahren}
\label{sec:ergebnisse_des_LDA}

\section{Visualisierung der Topic Models}
\label{sec:visualisierung_topicmodels}

\subsection{Topic Models als Heatmap}
\label{sec:heatmap}

\subsection{Topic Models als Netzwerk}
\label{sec:networks}

\section{Erstellung eines Topic Browsers}
\label{sec:erstellung_TopicBrowser}

\section{Visualisierung mit LDAvis}
\label{sec:anwendung_LDAvis}