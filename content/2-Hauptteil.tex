\chapter{Grundlagen}
\label{sec:Grundlagen}
Hier soll jeweils eine kurze Einf�hrung erfolgen, die den Zusammenhang des Kapitels zur Arbeit herstellt.

Generelle Hinweise: Verwenden Sie stets eindeutige Begrifflichkeiten achten Sie auf eine logische Herleitung ihrer  Argumentationen.

\section{Topic Models}
\label{sec:Topic Models}
\subsection{Latent Semantic Indexing}
\label{sec:LSI}

\subsection{Probabilistic Latent Semantic Analysis}
\label{sec:pLSA}

\subsection{Latent Dirichlet Allocation}
\label{sec:LDA}

\section{Darstellungsverfahren}
\label{sec:Darstellungsverfahren}

\subsection{Darstellung mit Standardbibliotheken}
\label{sec:darstellung_mit_standardbibliotheken}

\subsection{Topic Browser}
\label{sec:Topic Browser}

\subsection{LDAvis}
\label{sec:LDAvis}

\subsection{Termite}
\label{sec:Termite}

\chapter{Visualisierung der Ergebnisse}
\label{sec:visualisierung_der_ergebnisse}

\section{Darstellung des Outputs des LDA Verfahrens}
\label{sec:darstellung_des_outputs}

\section{Einfache Visualisierung des Outputs mit Standardbibliotheken}
\label{sec:einfache_visualisierung}


\subsubsection{Vierte Ebene}
\label{sec:vierte_ebene}

\paragraph{F�nfte Ebene}
\label{sec:f�nfte_ebene}


