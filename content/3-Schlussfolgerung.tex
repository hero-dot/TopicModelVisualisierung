\chapter{Schlussfolgerung}
\label{sec:schlussfolgerung}

In der Schlussfolgerung sollten folgende Punkte deutlich werden:
\begin{itemize}
	\item die Themenstellung
	\item	der gew�hlte Ansatz
	\item	die Eregbnisse der Arbeit
	\item eine kritische Stellungnahme/Einsch�tzung
	\item n�chste Schritte
\end{itemize}

Hinweis: Die Schlussfolgerung sollte mit der Zusammenfassung bzw. dem Abstract und der Einleitung abgeglichen werden. Es sollte immer eine Zusammenfassung der wesentlichen Erkenntnisse der eigenen Arbeit sein, die den Forschungsbeitrag darstellt. Der Umfang der Schlussfolge-rung sollte �hnlich wie die Einleitung ca. 5\% der gesamten Arbeit betragen.