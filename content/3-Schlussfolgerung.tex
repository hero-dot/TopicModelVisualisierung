\chapter{Schlussfolgerung}
\label{sec:schlussfolgerung}

Die Visualisierung von Topic Models kann mit verschiedenen Verfahren realisiert werden. Um die Problemstellung bei der Visualisierung und die Eignung der verschiedenen Verfahren zu �berpr�fen, wurden das LDA Modell und seine Dimensionen beschrieben. Im Anschluss wurden Verfahren beschrieben mit denen sich die Dimensionen des LDA Modells darstellen lassen. Dabei wurde auf statische Grafiken, wie Heatmaps und Netzwerke und auf interaktive Darstellungen, wie das LDAvis System und den Topic Browser eingegangen. Die Erstellung eines Topic Models zu dem 20Newsgroup Datensatz und die Visualisierung des Topic Models mit einer Heatmap und dem LDAvis Verfahren ergab die folgende Einsichten in die Problemstellungen bei der Visualisierung von Topic Models:
\\Die Visualisierung mit einer Heatmap eignet sich nur f�r Topic Models von kleinen Sammlungen an Dokumenten oder f�r die Ausschnittsweise Darstellung gr��erer Topic Models. Netzewerke werden mit zunehmender Gr��e des Topic Models  un�bersichtlich. Aufgrund der F�lle an Informationen die ein Topic Model bereith�lt ist eine interaktive Darstellung n�tig. Die Wahl der richtigen interaktiven Darstellung h�ngt von dem Ziel, das mit dem Erstellen des Topic Models verfolgt wird, ab. Ist das Ziel eine gro�e Sammlung an Texten �ber ihre latenten Themen zug�nglich zu machen, eignet sich daf�r ein Topic Browser. Soll die thematische Struktur eines Korpus analysiert werden, bietet sich die Verwendung des LDAvis Systems an. 
\\Die Qualit�t des verwendeten Vokabulars entscheidet �ber die Qualit�t des Topic Models und damit auch �ber die Aussagekraft der Visualisierung. Im vorliegendem Fall enthielten die Topics noch W�rter ohne Aussagekraft. Das f�hrt zu ungenauen Topics. Betrachtet man nun ein Kluster an Topics in der Darstellung mit dem LDAvis System, kann dem Kluster kein eindeutiger Themenbereich zugeteilt werden. 
Die Abbildung aller hier beschriebenen Verfahren wird f�r gro�e Topic Modelle un�bersichtlich. F�r die Heatmap ist das im Vergleich schon bei einer relativ geringen Anzahl an Topics der Fall. Das Netzwerk und das LDAvis System kommen mit einer gr��eren Anzahl an Topics zurecht. Aufgrund der Unterteilung des Topic Browsers in mehrere Ansichten und dem Abbilden einer Teilmenge an Topics und W�rtern kommt der Topic Browser auch mit sehr gro�en Dokumenten zurecht. 

%Anwendung weiterer Verfahren, die hier nur am Rande angesprochen worden sind
%Verwendung von Metadaten

