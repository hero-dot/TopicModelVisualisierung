\chapter{Zusammenfassung}
\label{sec:zusammenfassung}

Topic Models besitzen die F�higkeit latente Strukturen in Textsammlungen aufzudecken. F�r die Erstellung des Topic Models k�nnen verschiedene Modelle verwendet werden. Das am weitesten verbreitete ist das Latent Dirichlet Allocation Modell. Topic Models, die mit diesem Verfahren bestimmt wurden liegen als numerische Verteilungen vor. Diese Matrizen von Werten lassen sich durch ablesen nur schwer interpretieren. Eine visuelle Darstellung kann die Interpretation deutlich erleichtern. In dieser Arbeit werden Verfahren vorgestellt mit den Topic Models visuell dargestellt werden. Diese Verfahren sind unterteilt statische und interaktive Darstellungen. Bei den statischen Verfahren werden die Heatmap und ein Netzwerk vorgestellt. Die interaktiven Verfahren umfassen das LDAvis System und die Aufbereitung des Topic Models in Form eines Topic Browsers. 
Im Anwendungsteil der Arbeit wird ein Topic Model dem 20Newsgroup Datensatz erstellt. Bei dem Datensatz handelt es sich um einen sehr beliebten Trainingskorpus f�r Verfahren der maschinellen Textverarbeitung. Das erstellte Topic Model wird anschlie�end mit einer Heatmap und dem LDAvis System dargestellt. Mit den Darstellungen kann eine Interpretation der Topic Models erleichtert werden. Jedoch werden die Visualisierungen f�r gro�e Topic Models un�bersichtlich.